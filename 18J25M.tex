\chapter{18J25M}
\section{Charla de Bienvenida}

\section{Motivaci\'{o}n y Compromiso}

\section{Manual e Indicaciones}

\section{Aritm\'{e}tica en N\'{u}meros Reales}
\begin{enumerate}
  \item Establish the following properties of proportions: if $a$, $b$, $c$ and $d$ are nonzero reals, such that $\frac{a}{b}= \frac{c}{d}$, then
  \begin{equation*}
    \frac{a}{b}= \frac{c}{d}= \frac{a\pm c}{b\pm d}.
  \end{equation*}
  \begin{proof}
    \begin{eqnarray*}
      ab&=& ba,\\
      ab\pm bc &=& ba\pm bc,\\
      ab\pm ad&\overset{ad= bc}{=}& b(a\pm c),\\
      a(b\pm d)&=& b(a\pm c),\\
      \frac{a}{b}&=& \frac{a\pm c}{b\pm d}
    \end{eqnarray*}
  \end{proof}
  
  \item Give a sequence $(a_{1}, a_{2}, a_{3}, \dots)$ of digits, prove that the real number $0.a_{1}a_{2}a_{3}\dots$ represents the decimal expansion of a rational if and only if the sequence $(a_{1}, a_{2}, a_{3}, \dots)$ is periodic from some point on, in the sence of \ref{eq_1.3}.
  \begin{equation}\label{eq_1.3}
    (a_{1}, a_{2}, \dots, a_{l}, b_{1}, b_{2}, \dots, b_{p}, b_{1}, b_{2}, \dots, b_{p}, b_{1}, b_{2}, \dots, b_{p}, \dots)
  \end{equation}
  \begin{proof}
  $$\dots$$
  \hline
  \end{proof}
  
  \item
\end{enumerate}

\section{Relaci\'{o}n de Orden en los N\'{u}meros Reales}

\section{Completitud del sistema de n\'{u}meros reales}
